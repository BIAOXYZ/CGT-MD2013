\documentclass[12pt]{article}
\usepackage{amsfonts,amsmath,amssymb,graphicx,url}

% Old Stuff
%%\oddsidemargin=0.15in
%%\evensidemargin=0.15in
%%\topmargin=-.5in
%%\textheight=9in
%%\textwidth=6.25in

\setlength{\oddsidemargin}{.25in}
\setlength{\evensidemargin}{.25in}
\setlength{\textwidth}{6.25in}
\setlength{\topmargin}{-0.4in}
\setlength{\textheight}{8.5in}

\newcommand{\heading}[5]{
   \renewcommand{\thepage}{#1-\arabic{page}}
   \noindent
   \begin{center}
   \framebox{
      \vbox{
    \hbox to 6.2in { {\bf CS390 Computational Game Theory and Mechanism Design}
         \hfill #2 }
       \vspace{4mm}
       \hbox to 6.2in { {\Large \hfill #5  \hfill} }
       \vspace{2mm}
       \hbox to 6.2in { {\it #3 \hfill #4} }
      }
   }
   \end{center}
   \vspace*{4mm}
}

\newcommand{\handout}[3]{\heading{#1}{#2}{2012 ACM Class 5120309027 Huang Zen}{}{#3}}

\setlength{\parindent}{0in}
\setlength{\parskip}{0.1in}

\newenvironment{proof}{\noindent{\em Proof.} \hspace*{1mm}}{
\hspace*{\fill} $\Box$ }

\begin{document}
\handout{1}{July 8, 2013}{Problem Set 2}

\paragraph{Problem 1} (No collaborator.)
\\
Let $p$ be the probability player 1 chooses Up, and $q$ be the
probability player 2 chooses Left. Thus the four cases of strategy of
the two players would be with probability $pq$, $p(1-q)$, $(1-p)q$ and
$(1-p)(1-q)$.
\\
Then we have $u_A=9pq$, $u_B=9p+9q-18pq$, $u_C=9-9p-9q+9pq$,
$u_D=6-6p-6q+12pq$. There's no solution for $\{u_A\leq u_D, u_B\leq
  u_D, u_C\leq u_D\}$ with $p,q\in [0,1]$, so D is not a best response
for player 3.
\\
However there's also no solution for $\{u_A>u_D, u_B>u_D,
  u_C>u_D\}$ with $p,q\in [0,1]$, so D is not strictly dominated.
\bigskip

\paragraph{Problem 2} (No collaborator.)
\\
Consider a one-player game. The player will have two choices namely
$y$\&$n$ each time. When he chooses $n$ the game ends and he gets nothing;
if he chooses $y$ the game continues to the next choices between $y$ and $n$
so if he continously chooses $y$, the game will come to an infinite
history and give that history a positive utility.
\\
Now let's consider one strategy that on every occasion the player
chooses $n$. Obviously the player always gets nothing and the strategy
is not a SPE, even not a NE. However it does follow the condition in
one deviation property since that when the only deviation happens, the player just jumps to another
choice and chooses $n$ finally.

\bigskip

\paragraph{Problem 3} (No collaborator.)
\\
Notice that any pirate will always vote for himself. And if one cannot
get more coins through voting, he will not vote for the current pirate.
\\
First consider the situation that there are only two pirates, say
$D$ and $E$, $D$ takes all coins. 
\\
Then three pirates, say $C$, $D$ and $E$. If $C$ is thrown overboard $D$ can get all the
coins, so $D$ will not vote. If $E$ always gets nothing, $E$ prefers to throw $C$ overboard, so $C$ will
give $E$ one coin and himself 99 coins.
\\
Then four pirates, say $B$ to $E$. $C$, $D$ and $E$
will vote only if they get at least 100, 1 and 2 coin(s). So $B$ will give $D$ 1 coin and himself 99 coins.
\\
Finally five pirates. $B$, $C$, $D$ and $E$ will
vote only if they get at least 100, 1, 2 and 1 coin(s). So $A$
will give himself 98 coins and give $C$ and $E$ both 1 coin.
\\
The final distribution of coins will be $(98, 0, 1, 0, 1)$.

\bibliographystyle{agsm}

\begin{thebibliography}{99}

\bibitem{OR94}{M. J. Osborne and A. Rubinstein. {\em A course in game theory.} MIT Press, 1994.}

\bibitem{NRTV07}{N. Nisan, T. Roughgarden, E. Tardos, and V. Vazirani (eds). {\em Algorithmic game theory.} Cambridge University Press, 2007. (Available at \url{http://www.cambridge.org/journals/nisan/downloads/Nisan_Non-printable.pdf}.)}

\end{thebibliography}

\end{document}








