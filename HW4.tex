\documentclass[12pt]{article}
\usepackage{amsfonts,amsmath,amssymb,graphicx,url}

% Old Stuff
%%\oddsidemargin=0.15in
%%\evensidemargin=0.15in
%%\topmargin=-.5in
%%\textheight=9in
%%\textwidth=6.25in

\setlength{\oddsidemargin}{.25in}
\setlength{\evensidemargin}{.25in}
\setlength{\textwidth}{6.25in}
\setlength{\topmargin}{-0.4in}
\setlength{\textheight}{8.5in}

\newcommand{\heading}[5]{
   \renewcommand{\thepage}{#1-\arabic{page}}
   \noindent
   \begin{center}
   \framebox{
      \vbox{
    \hbox to 6.2in { {\bf CS390 Computational Game Theory and Mechanism Design}
         \hfill #2 }
       \vspace{4mm}
       \hbox to 6.2in { {\Large \hfill #5  \hfill} }
       \vspace{2mm}
       \hbox to 6.2in { {\it #3 \hfill #4} }
      }
   }
   \end{center}
   \vspace*{4mm}
}

\newcommand{\handout}[3]{\heading{#1}{#2}{2012 ACM Class 5120309027 Huang Zen}{}{#3}}

\setlength{\parindent}{0in}
\setlength{\parskip}{0.1in}

\newenvironment{proof}{\noindent{\em Proof.} \hspace*{1mm}}{
\hspace*{\fill} $\Box$ }

\begin{document}
\handout{1}{July 15, 2013}{Problem Set 4}

\paragraph{Problem 1} (No collaborator.)
\\
obviously $\sigma_i=\frac{1}{2}H+\frac{1}{2}T$ for each player $i$ is a Nash. 
In which way,
each player makes the other indifferent between choosing heads or tails,
so neither player has an incentive to try another strategy.
\\
Now given positive $\epsilon$, lets consider a strategy namely $\sigma'$:
$$
\sigma'=\left\{\begin{array}{ll}
\displaystyle(\frac{1}{2}+\frac{\epsilon}{20})H+(\frac{1}{2}-\frac{\epsilon}{20})T & i=1,...,5\\ \\
\displaystyle \frac{1}{2}H+\frac{1}{2}T &i=6	
\end{array} \right .
$$
We found that
\begin{align*}
&u_i(\sigma)=0 \tag{$i=1,...6$}\\
&|u_i(H,\sigma'_6)|=|u_i(T,\sigma_6)|=0\leq \epsilon \tag{$i=1,...,5$}\\
&|u_6((\sigma'_i),H)|=|5((\frac{1}{2}-\frac{\epsilon}{20})-(\frac{1}{2}+\frac{\epsilon}{20}))|
	=|-\frac{\epsilon}{2}|\leq \epsilon\\
&|u_6((\sigma'_i),T)|=|5((\frac{1}{2}+\frac{\epsilon}{20})-(\frac{1}{2}-\frac{\epsilon}{20}))|
	=|\frac{\epsilon}{2}|\leq \epsilon
\end{align*}
However $\sigma'$ is not a Nash since each player can always avoid receiving negative utility,
so $\sigma'$ yields a $\epsilon$-Nash equillibrium.

\bigskip

\paragraph{Problem 2} (No collaborator.)
\
W.L.G, 
take player 6 as the root and all other players be the children and terminated nodes.
\\
suppose player $6$ choose the strategy $\sigma = pH + qT$, $0 \leq p,q \leq 1$,$p+q=1$.
Player i chooses strategy $(p_iH,q_iH)$.
If $p=q$, any strategy for player $i$ is a best response.
If $p>q$, best response for other players would be $(1H,0T)$,
If $p<q$, the best response would be $(0H,1T)$.
\\
So the  table passed from the children is 
$$
T_i(p_i H+q_i T,pH+qT)=\left\{\begin{array}{ll}
1 & p=q\mbox{ or }p>q, p_i=1\mbox{ or }p<q, q_i=1\\
0 & {otherwise}
\end{array}\right .
$$
We have 
$$u_6(p_iH+q_iH,pH+qH)=p\sum_{i=1}^{5}q_i+q\sum_{i=1}^{5}p_i=(1-2p)(2\sum_{i=1}^{5}p_i-5)$$
For player $6$, $p\not=q$ is obvious not a best response to his children's choices
since all other players can choose one pure strategy and drop player $6$'s utility to negative.
And when $p=q$, it's a best response only if $\sum_{i=1}^{5}p_i=\frac{5}{2}$.
So the table at the root is
$$
T_6(pH+qT)=\left\{\begin{array}{ll}
	1 & p=q\\
	0 & \mbox{otherwise}
\end{array}\right .
$$
while the witness list at the root is 
$$ \sum_{i=1}^{5}p_i=\frac{5}{2}$$
This yields NEs with one player chooses equal probability $H$ and $T$, others chooses equal sum of probability of $H$ and $T$.

\bigskip

\paragraph{Problem 3} (No collaborator.)\\
\begin{proof}
In mixed strategy $\sigma$ yielded from strategy set $S$, let $p_i$ be the probability of $\sigma_i$, let $p=\displaystyle \sum_{i=1}^k \lambda_i p_i$ be the convex combination of $p_i$.
\\
$\forall i, \forall b \in S_i, \forall b' \in S_i \quad T=\{s \ |s\in S, \ s_i=b\}$
\begin{align*}
\displaystyle
\mathbb{E}_{s \sim \sigma |_{s_i=b}} u_i(s)
&=\sum_{s \in T} \frac{p(s)}{\displaystyle \sum_{s' \in S} p(s')} u_i(s)\\
&=\sum_{s \in T} \frac{\displaystyle\sum_{j=1}^k \lambda_j p_j(s)}{\displaystyle \sum_{s' \in S} p(s')} u_i(s)\\
&=\displaystyle\sum_{j=1}^k \lambda_j \frac{\displaystyle \sum_{s' \in S} p_j(s')}{\displaystyle \sum_{s' \in S} p(s')} \sum_{s \in T}  \frac{p_j(s)}{\displaystyle \sum_{s' \in S} p_j(s')} u_i(s)\\
&=\displaystyle\sum_{j=1}^k \lambda_j \frac{\displaystyle \sum_{s' \in S} p_j(s')}{\displaystyle \sum_{s' \in S} p(s')} \mathbb{E}_{s \sim \sigma_j |_{s_i=b}} u_i(s)\\
&\ge \displaystyle\sum_{j=1}^k \lambda_j \frac{\displaystyle \sum_{s' \in S} p_j(s')}{\displaystyle \sum_{s' \in S} p(s')} \mathbb{E}_{s \sim \sigma_j |_{s_i=b}} u_i(b', s_{-i})\\
&=\mathbb{E}_{s \sim \sigma |_{s_i=b}} u_i(b', s_{-i}) \mbox{ (by symmetry)}
\end{align*}
So
$$\mathbb{E}_{s \sim \sigma |_{s_i=b}} u_i(s) \ge \mathbb{E}_{s \sim \sigma |_{s_i=b}} u_i(b', s)$$

So for any finite game $G$, any finite convex combination of CEs is still a CE of $G$.
\end{proof}


\bibliographystyle{agsm}

\begin{thebibliography}{99}

\bibitem{OR94}{M. J. Osborne and A. Rubinstein. {\em A course in game theory.} MIT Press, 1994.}

\bibitem{NRTV07}{N. Nisan, T. Roughgarden, E. Tardos, and V. Vazirani (eds). {\em Algorithmic game theory.} Cambridge University Press, 2007. (Available at \url{http://www.cambridge.org/journals/nisan/downloads/Nisan_Non-printable.pdf}.)}

\end{thebibliography}

\end{document}








